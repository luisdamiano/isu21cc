\specialchapt{ABSTRACT}

Yield monitor datasets are known to contain a high percentage of
unreliable records. The current tool set is mostly limited to
observation cleaning procedures based on heuristic or
empirically-motivated statistical rules for extreme value
identification and removal. We propose a constructive algorithm for
processing well-documented yield monitor data artifacts without
resorting to data deletion. The RITAS algorithm models sample
observations as overlapping, unequally-shaped, irregularly-sized, and
time-ordered spatial units to better replicate the nature of the
destructive sampling process. Smoothing via a Gaussian Process is used
to provide map users with spatial-trend visualization. The
intermediate steps as well as the algorithm output are illustrated in
grain yield maps for an agricultural site.

%%% Local Variables:
%%% mode: latex
%%% TeX-master: "../thesis"
%%% End:
