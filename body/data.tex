\chapter{Data collection}

We first consider previous agronomical studies provide insight on the
physicalities of the natural system whose sampling process we aim at
replicating with our statistical algorithm. \cite{Ross2008} is a
synoptic work focused on yield estimation from yield monitor data
covering conceptual, modeling, and mechanical
aspects. \cite{Arslan2002} investigates the impact of yield monitor
calibration in yield estimation accuracy. Although not statistical in
nature.

We define crop yield as crop grain mass per unit area. Yield
monitoring equipment, introduced in the early 1990s, is a set of
typically more than six sensors, antennas, and devices attached to
combine harvesters that altogether calculate and record grain yield in
real time as the machine operator harvests the productive field. At
the end of each cycle, which lasts a pre-specified number of seconds
that can be adjusted dynamically by the operator, the monitor logs
measurements for more than 15 variables related to geolocation, crop
characteristics, and other machine diagnostics. The quantities

Relying on several on-board sensors such as yield monitors generate an
extremely large amount of observations.

As a combine harvester passes through a field, yield monitors acquire
almost in real-time multiple yield measurements all over the field. At
the same time, those data are associated with the GNSS positioning of
the machinery which enables precise location of each one of these
observations at the within-field level. As such, thousands of yield
spatial observations are generated and are ready to be used in the
decision-making process. 

Grain yield mapping using geo-referenced measurements from a
combine-mounted grain yield monitor are prevalent in
agriculture. Reported accuracy of continuous yield monitoring ranges
from 93 to 99.5 (Birrell et al. 1996; Arslan and Colvin 2002a; Fulton
et al. 2009) but depends on the type and brand of yield monitor,
calibration regime, flow rate and environmental conditions at harvest.

A combine's receiver estimates the position of the GPS
antenna. Receivers of this type currently in use achieve a root mean
square horizontal uncertainty between 0.5 and 3 m. Measurement of the
combine's position, however, is often less accurate than measurement
of the combine's velocity,

%%%%%%%%%%%%%%%%%%%%%%%%%%%%%%%%%%%%%%%%%%%%%%%%%%%%%%%%%%%%%%%%%%%%%%

Plant mass and harvested mass differ. it undergoes a series of
processes to separate grain from other materials and send clean grain
to the grain tank. operations of the combine: gathering, cutting,
pickup, and feeding, threshing, separation, and cleaning Grain loss
can be further broken down into “gathering loss” (grain missed or
dropped by the header), “processing loss” (grain lost due to standard
internal combine processes), and “leakage loss” (any combine losses
not due to gathering or standard internal processes) equal to the mass
of crop brought into the combine at that time note time dependence in
both gathering width and distance traveled. Amount of grain at a
specified location in the field is measured after the crop is
processed, usually on or at the exit of the clean grain
elevator. Thus, there is a time delay in flow rate measurements, which
addresses the time for grain to be transported from the combine head
to the yield sensor.

% The effective width is determined by the amount of underlap or overlap
% of adjacent planter passes the stationary mass, mG(tk), is transformed
% into a mass flow,

% confounding error conditions which must be screened out of
% the data.

% These datasets are effectively acquired through a variety of
% acquisition systems—machines, sensors—and on multiple crops, with
% different operators and under varying conditions of acquisition,
% e.g. topography or climate.

% n agriculture, data acquisition with on-board sensors can be
% understood as a sequential procedure through time during which a
% machine acquires information of a variable Z in space. Indeed, the
% data collection process follows a temporal dynamic,

% Birrell, S. J., Sudduth, K. A., & Borgelt, S. C. (1996). Comparison of sensors and techniques for crop yield mapping. Computers and Electronics in Agriculture, 14(2–3), 215–233.

% Arslan, S., & Colvin, T. S. (2002a). An evaluation of the response of yield monitors and combines to varying yields. Precision Agriculture, 3, 107–122.

% Fulton, J. P., Sobolik, C. J., Shearer, S. A., Higgins, S. F., & Burks, T. F. (2009). Grain yield monitor flow sensor accuracy for simulated varying field slopes. Applied Engineering in Agriculture, 25(1), 15–21.


%%% Local Variables:
%%% mode: latex
%%% TeX-master: "../thesis"
%%% End:
