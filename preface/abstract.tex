\specialchapt{ABSTRACT}

Yield monitor datasets are known to contain a high percentage of
unreliable records. The current tool set is mostly limited to
observation cleaning procedures based on heuristic or
empirically-motivated statistical rules for extreme value
identification and removal. We propose a constructive algorithm for
handling well-documented yield monitor data artifacts without
resorting to data deletion. The RITAS algorithm models sample
observations as overlapping, unequally-shaped, irregularly-sized,
time-ordered, areal spatial units to better replicate the nature of
the destructive sampling process. Observations are modeled as
rectangles. After their intersection area is assigned in time order,
area tessellation and harvested mass apportioning are applied to
generate regularly shaped and sized polygons. Finally, smoothing via a
Gaussian process is used to provide map users with spatial-trend
visualization. The intermediate steps as well as the algorithm output
are illustrated in maize and soybean grain yield maps across five
years for a research agricultural site located in the US Fish and
Wildlife Service Neal Smith National Wildlife Refuge.

%%% Local Variables:
%%% mode: latex
%%% TeX-master: "../thesis"
%%% End:
