\section{Introduction}

\cite{Miller1988} were the first to employ geostatistics explicitly in
precision agriculture, a discipline that started in 1990 when
agriculture shifted from farm-level to site-specific crop management
\citep{Oliver2010}. Their study, among other things, introduced
geostatistics for mapping patterns in soil phosphorus or potassium via
interpolation. As defined by the \citet{Council1997}, precision
agriculture is a data-centered discipline comprising data acquisition
at an appropriate scale, data interpretation and analysis, and
management response at an appropriate scale and time. The advent of
big data has been impacting this field largely, especially in terms of
data acquisition, interpretation, and analysis. Computer mapping of
yield and soil is one of the main uses of this technology, typically
to help customize crop management across and within fields by
identifying less productive areas at a sub-field scale
\citep{Lowenberg-DeBoer2019}. Yield data are now recorded
automatically for a wide variety of crops including cereal grains,
oilseeds, fiber, forage, biomass, fruits and vegetables. These data
are known to be accurate at a global scale, yet there exist nuances at
a local scale that affect visualization and downstream analysis, and
jeopardizes the credibility and validity of management decisions
downstream.

\cite{Ross2008} is a synoptic work focused on yield estimation from
yield monitor data covering conceptual, modeling, and mechanical
aspects. \cite{Arslan2002} investigates the impact of yield monitor
calibration in yield estimation accuracy. Although not statistical in
nature, agronomical studies provide with insight on the physicalities
of the system whose sampling process we aim at replicating with our
statistical algorithm. Yield monitoring equipment, introduced in the
early 1990s, is set of typically more than six sensors, antennas, and
devices attached to combine harvesters that altogether calculate and
record grain yield in real time as the machine operator harvests the
productive field. At the end of each cycle, which lasts a
pre-specified number of seconds that can be adjusted dynamically by
the operator, the monitor logs measurements for more than 15 variables
related to geolocation, crop characteristics, and other machine
diagnostics.

Numerous works have established that yield datasets contain a high
percentage of unreliable data and have proposed cleaning procedures
\citep{Blackmore1996, Moore1998, Blackmore1999, Thylen2000, Noack2003,
  Simbahan2004, Ping2005, Sudduth2007, Sudduth2012, Spekken2013,
  Leroux2018, Leroux2019, Vega2019}. Mechanical measurement errors
have been studied in great depth \citep{Arslan1999, Arslan2002,
  Grisso2002, Burks2004, Hemming2005, Fulton2009, Schuster2017}, yet
no model-based data adjustment protocols has been developed. Instead,
systematic errors are identified using either heuristics or
statistical rules with acceptable empirical results but no underlying
probabilistic justification. With almost unanimous preference for
systematic error detection and removal, little consideration has been
given to alternative strategies for error correction or integration
such as the one proposed by \cite{Bachmaier2007, Bachmaier2010}; in
fact, some methodologies discard as much as one third of the original
dataset as summarized by \cite{Lyle2013}.

In most cases, these procedures require the map user to set values for
tuning parameters such as thresholds. These constants are not learned
from the data but set arbitrarily by the user instead, hindering
comparisons across users, locations, and years. Moreover, manually-set
thresholds become a hurdle in the context of big data: both the amount
and the heterogeneity of yield monitor datasets have been increasing
as raw data become more abundant and diverse. Additionally, there is a
general preference for working with crop yield as the main input of
the cleaning procedures. Being a variable that consolidates data
collected from more than one sensor, each with its own propagating
measurement error, it has a lower signal-to-noise ratio than the mass
random variable directly. Also, scaling mass to homogenize
unequally-sized observational units potentiates extreme values
resulting in a new variable with higher volatility.

Some of the processing rules, for example those based on the marginal
distribution of yield, do not consider the spatial nature of the data
\LD{I have some works to cite, but I don't think it's cool to single
  out these cases.}. Those that do simply collapse all the information
in one point on a 2-dimensional plane, thus failing to recognize that
the recorded data are in fact associated with overlapping,
unequally-shaped, and irregularly-sized aerial units. Each observation
is in reality a discrete sample from a continuous spatial stochastic
process: time order and spatial superposition are intrinsic
characteristics of the destructive sampling scheme worth modeling.

The many variables logged by the yield monitor equipment include
harvested mass, geocordinates, distance traveled, and swath width,
which can be used to create an aerial representation of the
data. Depending on the hardware, the reported travel distance and
speed may be measured by a speed sensor, a GPS receiver, a radar, or a
ultranosic sensor \citep{Mulla2013}. When distance is not available,
it can be linearly estimated using speed and cycle length or
approximated using the euclidean distance between two subsequent
coordinates. The swath width, also known as gathering or cutting
width, has a complex dependence with respect to time and the combine's
path that may not be reflected in the data logged by the monitor
\citep{Ross2008}. Although equipments provides the operator with tools
to adjust the effective width dynamically, in many situations the
operator is unable to attend to manual adjustments in real time and
thus the logged values simply represents the maximum width. We have
found three main strategies that have been introduced so far that
combine distance and width to reproduce an approximate spatial
representation.

\cite{Blackmore1999} observed that there is a discrepancy between the
theoretical and the effective harvested area. One of the reasons is
that the recorded header width differs from the width of the header
that is truly full of crop. While in practice it can be set equal to a
fixed proportion of the cutter bar width, e.g. 95\%, land finishes and
areas close to voids require special treatment as header could even be
empty. In order to avoid any use of the recorded width, and circumvent
the uncertainty associated with it, the authors introduced potential
mapping. In this technique, the recorded mass of data points belonging
to circular neighbourhoods is aggregated and assigned to a new spatial
point whose location corresponds to the circle geocenter. The
aggregated mass is then spatially re-normalized by the area of the
polygons in the Voronoi diagram formed by these new points. This
approach has two limitations. First, it transforms data into spatial
points and thus incurs in information loss about the shape and size of
the observational aerial units (e.g. different shapes could have the
same centroid). Second, a gridding technique needs to be applied to
analyze temporal trends in yield maps \citep{Blackmore2003},
effectively displacing the yield measurements twice: all the polygon
information is first collapsed into a single point which is displaced
afterwards.

Previously, \cite{Han1997} introduced a bitmap approach to determine
the actual combine cut width, compute the effective area size, and
approximate the new centroid of each spatial unit after removing the
area covered by previous observations. To track the covered area at
each time step, the author superposes a regular grid where each cell
functions as a bitmask indicating whether the pixel has been harvested
before or not, hence the name of the methodology. The article does not
show how this methodology performs in the presence of sharp turns,
where overlaps tend to be larger during the pivoting motion, as these
seem to have been removed from the analysis. It is worth noting that,
even after processing, the intermediate spatial representations have
some overlap and skips that the authors leave unexplained. Again, as
these intermediate spatial representations are collapsed into spatial
points, some information is lost. Furthermore, these new data points
may not be spationally aligned if temporal analysis was pursued
next. Finally, the authors warn that the bitmap initialization process
is more complicated when there exists non-crop features such as
rivers, canals, or roadways.

Expanding on the bitmap concept, \cite{Drummond1999} developed a
method where observational units are represented by polygons
constructed from position and trajectory information. These are
processed in chronological harvest order by Boolean subtraction to
compute the actual harvested area during each time step, effectively
retaining full information about the aerial units. As drawbacks, the
authors mention the potential overestimation at the boundaries as well
as the computational complexity of the algorithm. In the general case,
its complexity is order $O(N^2)$, yet several strategies for specific
cases are discussed there.

