\section{STRIPS1 Yield Maps}

\OUTLINE{Introduce section} In this section, we illustrate the
functioning and the results of our methodology when applied to yield
monitor data collected from the same agricultural site over the
years. We start with a detailed discussion of the production of the
yield map for one specific year, and we finally show some resulting
visualizations for the same fields but different crops and years.

\OUTLINE{Introduce data context} The data arises from a study
conducted at the Neal Smith National Wildlife Refuge in central Iowa
to quantify the impact of grassland-to-cropland conversion on
nitrate-nitrogen (NO\textsubscript{3}–N) concentrations in soil and
shallow groundwater and to assess the potential for perennial filter
strips to mitigate increases in NO\textsubscript{3}–N levels
\citep{Zhou2010}. The experiment was run in different study sites
within the 3000-ha area managed by the U.S. National Fish and Wildlife
Service, located in the Walnut Creek watershed in Jasper County,
Iowa. In this case study, we focus on one specific site named Basswood
that is situated west to the Basswood Trailhead.

\OUTLINE{Introduce data specifics} Basswood, located at WGS84 15 N
0477097E 4598644N, has a total area of approximately 13 Ha. Nearly
81\% of the surface is cropland; most of the remaining proportion is
reconstructed prairie vegetation planted as part of the experimental
design. For the purpose of our yield maps, these areas are treated as
voids because no data was collected from that surface. Cropland in the
experiment is in a corn–soybean rotation using standard no-till soil
and weed-management techniques. Geographic coordinates, sample time,
moisture content, and corn (2008, 2010, 2012, 2014) and soybean (2009,
2011, 2013, 2015) flow rate were reported by a Case IH AFS Pro-600
combine-mounted yield monitor every 1-3 s during crop harvest,
resulting in a fine-scale spatially referenced dataset of crop yields
across the study area. \cite{Schulte2017} provides a detailed account
of the experiment protocol and the resulting improvements in the
biodiversity and the delivery of multiple ecosystem services.

\OUTLINE{Describe data} The yield monitor dataset for the year 2012
has 4,239 observations logged each three seconds starting from the
southwest corner of the site. The swath width was reported to be
constant at 6.10 m as the operator did not adjust the proportion of
cut head while driving. The distance traveled during each cycle, with
an overall mean of 3.7 m, has three modes with centers at 1.9, 4.0,
and 6.0 m. The distribution of the yield reported by the monitor, with
its median located at 6.3 mg/ha and the 90\% of observations being
within 1.4 and 10.8 mg/ha, is symmetric and platykurtic. Visual
inspection suggests that there are approximately 20 extreme values on
the right tail. Small areas within the field boundaries without data,
visualized as voids in the maps, correspond to small portions of soil
allocated for nonproductive purposes (e.g. perennial crops, research
equipment).

\begin{figure}[h!]  \centering
  \includegraphics[width=\textwidth]{basswood_2012_res5_points_vehicle}
  \includegraphics[width=\textwidth]{basswood_2012_res5_1_agg_smoothed}
  \caption{Visualization of some of the intermediate steps involved
    in the algorithm for the harvested grain yield of Basswood on year
    2012 (corn). \underline{Top left}: Dot map where each observation is
    marked with equally-sized points placed at the each logging
    location. Information relevant to spatial trends are hard to observe,
    such as area coverage and overlaps. \underline{Top right}: Map of the
    clipped polygons. Highly intertiment coloring in noisy areas hinder
    the visualization of the spatial trends, hence the need for
    smoothing. \underline{Bottom left}: Map of the aggregated grid at a 5
    m resolution. This step produces equally-sized, regularly-shaped
    polygons suitable for spatial interpolation. \underline{Bottom right}:
    Map to be shown to the user. By increasing color homogeneity at a
    local level, low and high yield areas have a larger contrast and
    become easier to identify.}
  \label{fig:basswood2012-main-steps}
\end{figure}

\OUTLINE{Introduce dot maps} Figure \ref{fig:basswood2012-main-steps}
displays the simplest form of a yield map. Data points are visualized
as equally-sized symbols with colors indicating the yield level at
each location. A single-hue shade of green is chosen to ease the
interpretation with dark shades being intuitively associated with
relatively higher crop yield. The characteristics of the yield
distribution vary largely across sites, crop types, and years. Since
the main objective of yield map analysis is to identify sub-field
areas with different performance levels, the dots are colored
according to a measure of relative yield within the dataset as opposed
to an absolute scale. As a general rule, we define the color gradient
as a function of the the empirical quartiles in order to guarantee
that low and high yield measurements are uniformly represented at the
same time that the interpretation guidelines remain consistent across
maps.

\OUTLINE{Describe dot map} The proposed coloring scheme helps
capturing the large-scale spatial trends: the east area, both above
and below the perennial crop strip, and the northwest area show the
best relative performance whereas the southwest borderline is
underperforming. Another evident pattern is outer borders and borders
around inner voids displaying lower yield, which could be explained by
soil fertility properties or could simply be a byproduct of how data
is collected. Data artifacts due to narrow finishes are well
documented: if the header cut was not full when harvesting the
boundaries and the operator failed to manually flag it, yield would be
underestimated. Careful consideration should be given to neighbouring
points that are not on the borders. Light green points on the northern
borderline are adjacent to dark green points, suggesting that this
might very well be an artifact due to deficiencies in the the data
collection procedure. On the other hand, data points on the southwest
zones are consistently underperforming, thus suggesting the existence
of an actual spatial trend.

\OUTLINE{Criticize dot map} Using points to visualize the data
provides no information about the shape of the aerial observational
unit and hides the overlaps in the harvested areas, which should be
considered appropriately when computing the estimated yield at a given
spatial location. In fact, from the figure it is not evident that
there is 9.0\% of aerial overlap, defined as the percent excess of the
sum of the individual rectangles area over the polygons union area.

\OUTLINE{Step 1: Create polygons} We apply the XYZ algorithm. The
shape of the site forces the operator to turn several times, including
some sharp turns. Figure \ref{fig:basswood2012-all-steps}, displaying
the constructed spatial polygons, makes overlapping patterns more
evident: (i) subsequent samples overlap during turns, especially on
the inner side; (ii) near voids, where the landscape requires more
maneuvering; (iii) wedges formed by perpendicular passing, for example
on the western part of the field; (iv) driving from one part to
another; (v) between parallel passings and narrow segments.

\OUTLINE{Step 2: Clipping} Because overlapping produces a systematic
overestimation of the effective area size, thus biasing down the
estimated yield, its correct treatment can uncover performing
areas. In all these cases, the coloring suggests that highly
overlapping polygons are associated with lower yields. We note,
however, that yield was computed using the theoretical polygon area
which overestimates the effective harvest area. In Figure
\ref{fig:basswood2012-all-steps}, which shows the reshaped polygons
and the yield computed with the new effective area, we notice that
some of the sub field areas with low-yield polygons now display a
better performance suggesting that this visual artifact was indeed
caused by the overlapping. As a computational note, when producing the
reshaped polygons, 10 spatial polygons whose area had been fully
harvested in previous time steps were dropped from the dataset causing
an minor leakage of 0.1\% of the total harvest mass; in case of major
situations, aggregating the mass of fully nested geometries would be
appropriate.

\OUTLINE{Motivate smoothing} As the ultimate goal of the visualization
work is to support the map user's decision making process, the clipped
map is not adequate. Crop management decisions at a sub-field scale
are based on spatial trends whereas the measurements are highly noisy
due to a combination of at least 10 possible types of data collection
error as discussed in \cite{Lyle2013}. For example, in Figure
\ref{fig:basswood2012-all-steps} there are zones of predominating high
and low yield contaminated with scattered observations with the
opposite color, and there are also zones with a combination where it
is difficult to identify local trends.

\OUTLINE{Steps 3 \& 4: Gridding and tiling.} Smoothing is thus
typically applied. As discussed in Section \ref{sec:algorithm},
unequally-sized polygons are not suitable for off-the-shelf smoothing
techniques and so we resource to tiling and aggregation. We
superimpose a grid with 4,194 squares at a 5-meter resolution and
apportion the harvested mass of the reshaped polygons into the
corresponding pixels. As we retain those pixels with at least 50\% of
their area covered by observations only, we find local zones with
under or over coverage. Overall, the whole grid covers 2.6\% less of
the sampled surface, a discrepancy that can be diminished by
increasing the grid resolution.

\OUTLINE{Step 5: smoothing} The results are seen in Figure
\ref{fig:basswood2012-main-steps}. The effect of smoothing on the
signal to noise ratio is evident from the figure. The richness of
smoothing is not only on the visualization, but also on the
interpretation of the estimated parameters. We then propose that yield
maps should not only include the spatial polygons, but also
statistical information useful for the map user. The range, for
instance, is informative for map users to better understand the scale
of the spatial effect and adjust the scale of their decisions
accordingly. % define the spatial scale of their management decisions.

\OUTLINE{Step 5: smoothing cnt'd} The smooth map is consistent with
the main spatial trends observed so far. Clear patterns become more
evident now; in the mixed areas, smoothing helps not only to identify
the overall local trend but also to make internal breaks/borders more
visible. Additional features can help with interpretation include
contour lines, e.g. each 1 mg/ha similar to \cite{Blackmore1999}, or
color schemes based on spatial clusters.

We have implemented our algorithm in the R programming language, using
the R packages doParallel, foreach, gstat, rgeos, and sp
\citep{Pebesma2004, Pebesma2005, Bivand2013, Graeler2016,
  Microsoft2017, Corporation2018, RCT2019, Bivand2019}. Our open-source
`yieldMaps` package contains routines for the algorithm described in
the article, visualization routines, and the dataset employed in the
case study. The algorithm routines were programmed as smaller, modular
tasks so that users could easily adapt the source code to custom needs
such as designing a different gridding and aggregation strategy, or
employing another spatial interpolation model. Visualization routines
include map generation as well as object movement and polygon creation
visualizations.

When doing exact inference for a Gaussian Process, smoothing requires
the inversion of a square matrix with size equal to the number of
observations in the aggregation step, which in this case study is
3,738 pixels. Unless resorting to approximate interpolation methods,
significant time is expected to be consumed for this purpose. As a
palliative, our implementation optionally divides the prediction space
into smaller subsets to be computed in parallel. Although there is no
direct gain in the matrix inversion time, it provides with some
marginal improvements as out prediction space is large with more than
90,000 pixels.

Using 10 cores, the total computational time of 201 m included 168 m
(83\%) and 17 m (8\%) spent in the smoothing and tiling steps
respectively. This suggests that the algorithm has be run offline for
any moderately-sized datasets, but could be improved vastly by
simplifying the spatial interpolation model. The resulting maps for
the same field but different years and crops are shown in Figure
\ref{fig:basswood-history}.

\begin{figure}[h!]  \centering
  \includegraphics[width=0.49\textwidth]{basswood_2010_res5_1_agg_smoothed}
  \includegraphics[width=0.49\textwidth]{basswood_2014_res5_1_agg_smoothed}
  \includegraphics[width=0.49\textwidth]{basswood_2009_res5_1_agg_smoothed}
  \includegraphics[width=0.49\textwidth]{basswood_2011_res5_1_agg_smoothed}
  \caption{\textbf{Fix the captions after settling for a
      layout}. Visualization of the harvested grain yield for Basswood on
    the years 2010 (corn), 2014 (corn), 2009 (soybean), and 2011 (soybean)
    respectively. Aggregated grid (left) and the smooth map
    (right). Although soybean data shows larger variability, as clearly
    displayed on the aggregated grid, the smooth map offer a clear
    visualization of the spatial trends.}
  \label{fig:basswood-history}
\end{figure}
