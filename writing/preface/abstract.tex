\specialchapt{ABSTRACT}

Yield monitor datasets are known to contain a high percentage of
unreliable records. The current tool set is mostly limited to
observation cleaning procedures based on heuristic or
empirically-motivated statistical rules for extreme value
identification and removal. We propose a constructive algorithm for
handling well-documented yield monitor data artifacts without
resorting to data deletion. The four-step Rectangle creation,
Intersection assignment and Tessellation, Apportioning, and Smoothing
(RITAS) algorithm models sample observations as overlapping,
unequally-shaped, irregularly-sized, time-ordered, areal spatial units
to better replicate the nature of the destructive sampling
process. Positional data is used to create rectangular areal spatial
units. Time-ordered intersecting area tessellation and harvested mass
apportioning generate regularly -shaped and \mbox{-sized} polygons
partitioning the entire harvested area. Finally, smoothing via a
Gaussian process is used to provide map users with spatial-trend
visualization. The intermediate steps as well as the algorithm output
are illustrated in maize and soybean grain yield maps for five years
of yield monitor data collected at a research agricultural site
located in the US Fish and Wildlife Service Neal Smith National
Wildlife Refuge.

%%% Local Variables:
%%% mode: latex
%%% TeX-master: "../thesis"
%%% End:
