\chapter{Data collection}

For our purposes, we define crop yield as crop grain mass per unit
area. Our quantity of interest is a ratio of two quantities that
involve a set of typically more than six sensors, antennas, and
devices attached to combine harvesters that altogether calculate and
record grain yield in real time as the machine operator harvests the
productive field. At the end of each cycle, which lasts a
pre-specified number of seconds, the monitor logs measurements from
several on-board sensors for more than 15 variables related to
geolocation, crop characteristics, and machine diagnostics. Dataset
size could range from one to more than a hundred thousand or more
observations depending on factors such as field size, cycle length,
and land features.

Grain yield mapping using geo-referenced measurements from a
combine-mounted grain yield monitor are prevalent in
agriculture. Reported accuracy of continuous yield monitoring ranges
93-99.5\% depending on equipment type and brand, calibration regime,
flow rate, and environmental conditions at harvest
\citep{birrellComparisonSensorsTechniques1996, Fulton2009,
  Lyle2013}. \cite{Arslan2002} investigates the specific impact of
yield monitor calibration in yield estimation accuracy.

Previous agronomical studies offer insight on the physicalities of the
natural system whose sampling process we replicate. \cite{Ross2008} is
a synoptic work focused on yield estimation from yield monitor data
covering conceptual, modeling, and mechanical aspects. Grain mass
measurement involves a series of operations to separate clean grain
from other materials: gathering, cutting, pickup, and feeding,
threshing, separation, and cleaning. As a result, there is a time
delay in flow rate measurement as well as several mass losses
associated with plant gathering, processing, and leaking until the
grain is measured on or at the exit of the clean grain elevator. At
the end of each cycle, a receiver estimates the GPS antenna position,
whose placement in the combine varies across models. Their
capabilities vary largely, typically being more accurate at measuring
velocity rather than position and having a root mean square horizontal
uncertainty of 0.5-3.

%%% Local Variables:
%%% mode: latex
%%% TeX-master: "../thesis"
%%% End:
